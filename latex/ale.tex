\documentclass{article}
%\documentclass{amsart}
%\documentclass{ws-m3as}
%\documentclass{m2an}
% SPRINGER
%\documentclass{svjour3}                    % onecolumn (standard format)
%\documentclass[smallextended]{svjour3}     % onecolumn (second format)
%\documentclass[twocolumn]{svjour3}         % twocolumn
%------------------------------------------------------------
% Defining page layout:
\pagestyle{plain} 
\setlength{\paperwidth}{210mm}
\setlength{\paperheight}{297mm} 
% on compte du bord du papier
\setlength{\hoffset}{-10mm}
\setlength{\voffset}{0mm} 
\setlength{\textwidth}{150mm}
\setlength{\textheight}{200mm} 
% marge gauche
\setlength{\evensidemargin}{0mm}
%\setlength\oddsidemargin{3cm}
% entete
\setlength{\topmargin}{0mm} 
%\setlength{\headheight}{0mm}
\setlength{\headsep}{10mm}
%pied de page
\setlength{\footskip}{20mm} 
% notes en marge droite
\setlength{\marginparsep}{0mm}
\setlength{\marginparwidth}{0mm} 
%\setlength\marginparpush{0cm}
% 
\parindent=0.5cm
\linespread{1.2}
% \textwidth 455pt \oddsidemargin 0pt \evensidemargin 0pt 
% \headsep 0pt \headheight 0pt \textheight 655pt \parskip 10pt
% \def\arraystretch{1.15}
% \renewcommand{\floatpagefraction}{0.75}
%------------------------------------------------------------
% Adding packages:
%\usepackage{url}
% mathematical symbols
\usepackage{latexsym}
\usepackage{amsmath}%
%\usepackage{amsfonts}%
\usepackage{amssymb}%
\usepackage{amsthm}
%\usepackage{dina4}
%\usepackage{mathtools} % (over/under)brackets
\usepackage{bm}
% graphical tools
\usepackage{graphicx}
%\usepackage{psfrag,psfig}
%\graphicspath{./fig/}
%\usepackage[pdftex]{graphicx}
%\usepackage{epsf}
%\usepackage{epsfig}
\usepackage[all]{xy}
\usepackage{hhline}
%\usepackage{exscale,cmmib57}
\usepackage{epic} %,eepic}% attention, dashline disparait avec eepic
%\usepackage{subfigure}
%\DeclareGraphicsExtensions{.jpg,.png,.pdf}
%\DeclareGraphicsRule{*}{mps}{*}{}
%\graphicspath{{fig/}}
% packages for layout
%\usepackage{geometry}
%\usepackage{afterpage}
%\usepackage{fullpage}
%\usepackage{fancyhdr}
% my additional package for fonts featuring
%\usepackage{mathptmx}      % SPRINGER use Times fonts if available on your TeX system
\usepackage[english]{babel}
%\usepackage[francais,english]{babel}
\usepackage[latin1]{inputenc}
\usepackage[T1]{fontenc}
%\usepackage[cyr]{aeguill} % guillemets
% my additional package for showing the labels
%\usepackage{showkeys}
%\usepackage[notref,notcite]{showkeys}
%\usepackage{showlabels}
%\usepackage{showtags}
%\usepackage{drafcopy}
%\usepackage{float}
\usepackage{enumerate} % to control the display of the enumeration counter
%------------------------------------------------------------
% Counters
%\newcounter{qn}[section]
%------------------------------------------------------------

%\smartqed  % SPRINGER flush right qed marks, e.g. at end of proof
% Theorem like environments
\newtheorem{theorem}{Theorem}
\theoremstyle{plain}
\newtheorem{acknowledgement}{Acknowledgement}
\newtheorem{algorithm}{Algorithm}
\newtheorem{axiom}{Axiom}
%\newtheorem{case}{Case}
%\newtheorem{claim}{Claim}
%\newtheorem{conclusion}{Conclusion}
%\newtheorem{condition}{Condition}
%\newtheorem{conjecture}{Conjecture}
\newtheorem{corollary}{Corollary}
%\newtheorem{criterion}{Criterion}
\newtheorem{definition}{Definition}
\newtheorem{example}{Example}
%\newtheorem{exercise}{Exercise}
\newtheorem{lemma}{Lemma}
\newtheorem{notation}{Notation}
\newtheorem{problem}{Problem}
\newtheorem{proposition}{Proposition}
\newtheorem{remark}{Remark}
%\newtheorem{solution}{Solution}
\newtheorem{summary}{Summary}
\numberwithin{equation}{section} % counter !!!!!!!
%\newtheorem{theorem}{Theorem}[section]
%\newtheorem{lemma}{Lemma}[section]
%\newtheorem{proposition}{Proposition}[section]
%\newtheorem{assumption}{Assumption}[section]
%\newtheorem{corollary}{Corollary}[section]
%\newtheorem{definition}{Definition}[section]
%\newtheorem{conjecture}{Conjecture}[section]
%\newtheorem{problem}{Problem}[section]
%\newtheorem{remark}{Remark}[section]
\newcommand\beq{\begin{equation}}
\newcommand\eeq{\end{equation}}
\renewcommand{\emph}{\textbf}
%\renewcommand{\emph}[1]{{\large \slshape #1}}
%------------------------------------------------------------
%------------------------------------------------------------
% Delimiters, norms, inner products...:
%
\newcommand{\brk}[1]{\left(#1\right)}          % \brk{.}     => (.)
\newcommand{\E}[1]{\mathbb{E}\brk{#1}}
\newcommand{\V}[1]{\mathbb{V}\brk{#1}}
\def\E{\mathbb{E}}
\def\V{\mathbb{V}}
\newcommand{\Brk}[1]{\left[#1\right]}          % \Brk{.}     => [.]
\newcommand{\BRK}[1]{\left\{#1\right\}}        % \BRK{.}     => {.}
\newcommand{\Average}[1]{\left<#1\right>}      % \Average{.} => <.>
\newcommand{\mean}[1]{\overline{#1}}           % \mean{.}
\newcommand{\Abs}[1]{\left| #1 \right|}        % \Abs{.}     => |.|
\newcommand{\Scal}[2]{\left(#1,#2\right)}      % \Scal{.}    => (.;.)
\newcommand{\Norm}[1]{\left\| #1 \right\|}     % \Norm{.}    => ||.||
\newcommand{\mymat}[1]{\begin{pmatrix} #1 \end{pmatrix}}
\newcommand{\jump}[1]{[\![#1]\!]}
%
%------------------------------------------------------------
% Special characters and shortcuts:
%
% my constants and parameters
\newcommand{\dd}{\frac{d(d+1)}{2}}
\newcommand{\dt}{\Delta t}
\newcommand{\Wi}{{\text{Wi}}}
\renewcommand{\Re}{{\text{Re}}}
\newcommand{\Ma}{{\text{Ma}}}
\newcommand{\Ra}{{\text{Ra}}}
\newcommand{\Fr}{{\text{Fr}}}
\renewcommand{\Pr}{{\text{Pr}}}
\newcommand{\Gr}{{\text{Gr}}}
\newcommand{\e}{\varepsilon}
\newcommand{\D}{\mathcal{D}}
\newcommand{\DD}{\mathrm{D}}
\newcommand{\f}{\boldsymbol{f}}
\newcommand{\I}{\boldsymbol{I}}
\newcommand{\xx}{\boldsymbol{x}}
\newcommand{\bolda}{\boldsymbol{a}}
\newcommand{\boldb}{\boldsymbol{b}}
\newcommand{\bzero}{{\boldsymbol{0}}}
\def\bn{\boldsymbol n}
\def\bN{\boldsymbol N}
\def\J{\boldsymbol J}
\newcommand{\NN}{\mathcal{N}}
% my limits
\renewcommand{\to}{\rightarrow}
\newcommand{\To}{\longrightarrow}
% my operators
\renewcommand{\div}{\operatorname{div}}
\newcommand{\curl}{\operatorname{curl}}
\newcommand{\tr}{\operatorname{tr}}
\newcommand{\range}{\operatorname{Range}}
\newcommand{\spn}{\operatorname{Span}}
\newcommand{\dist}{\operatorname{dist}}
\newcommand{\grad}{\boldsymbol{\nabla}}
\newcommand{\pd}[2]{\frac{\partial#1}{\partial#2}}
\newcommand{\deriv}[2]{\frac{d#1}{d#2}}
\newcommand{\pdd}[2]{\frac{\partial^2#1}{\partial#2^2}}
\newcommand{\pds}[1]{\partial_{#1}}
\newcommand{\pdds}[2]{\partial^2_{#1#2}}% my integrals
\newcommand{\intd}{\int_\D}
%\renewcommand{\liminf}[1]{\underset{#1}{\operatorname{liminf}}}
% my spaces
\newcommand{\N}{\mathbb{N}}
\newcommand{\R}{\mathbb{R}}
\newcommand{\RS}{\mathbb{R}^{d \times d}_S}
\newcommand{\RSPD}{\mathbb{R}^{d \times d}_{SPD}}
\newcommand{\PP}{\mathbb{P}}
\newcommand{\QQ}{\mathbb{Q}}
% \def\U{\mathrm{W}}  % !! \V does not always compile !!
% \def\Uz{\mathrm{V}}
\def\U{\mathrm{U}}
\def\V{\mathcal{V}}
\def\V{\mathrm{V}}
\def\H{\mathrm{H}}
\def\W{\mathrm{W}}
\def\Q{\mathrm{Q}}
\def\S{\mathrm{S}}
\def\SPD{\S_{PD}}
\def\Uh{\mathrm{W}_h}
\def\Uhz{\mathrm{V}_h}
\def\Vhzero{\mathrm{V}_{h}^0}
\def\Vhone{\mathrm{V}_{h}^1}
\def\Qh{\mathrm{Q}_h}
\def\Sh{\mathrm{S}_h}
\def\Shzero{\mathrm{S}_h^0}
\def\Shone{\mathrm{S}_h^1}
\def\ShonePD{\mathrm{S}_{h,PD}^1}
% my norms
\newcommand{\NormW}[3]{\|#1\|_{#2,#3}}
\newcommand{\NormH}[2]{\|#1\|_{#2}}
\newcommand{\NormLinf}[1]{\|#1\|_{L^\infty}}
\newcommand{\NormLone}[1]{\|#1\|_{L^1}}
\newcommand{\NormLtwo}[1]{\|#1\|_{L^2}}
\newcommand{\NormC}[1]{\|#1\|_{C^0}}
\newcommand{\NormHolder}[3]{\|#1\|_{C^{#2,#3}}}
\newcommand{\SemiNormHolder}[2]{|#1|_{C^{#2}}}
% my fluid variables
\newcommand{\str}{{\boldsymbol{\tau}}}
\newcommand{\strs}{{\boldsymbol{\sigma}}}
\newcommand{\bu}{{\boldsymbol{u}}}
\newcommand{\bv}{\boldsymbol{v}}
\newcommand{\bw}{\boldsymbol{w}}
\newcommand{\bz}{\boldsymbol{z}}
\newcommand{\be}{\boldsymbol{e}}
\newcommand{\bg}{\boldsymbol{g}}
\newcommand{\vort}{{\boldsymbol{\omega}}}
\newcommand{\gbu}{{\grad\bu}}
\newcommand{\gbv}{{\grad\bv}}
% \newcommand{\Dbu}{{\frac12(\gbu+\gbu^T)}}
% \newcommand{\Dbv}{{\frac12(\gbv+\gbv^T)}}
\def\Dbu{\boldsymbol{D}(\bu)}
\def\Dbv{\boldsymbol{D}(\bv)}
\newcommand{\bphi}{\boldsymbol{\phi}}
\newcommand{\bpsi}{\boldsymbol{\psi}}
\newcommand{\bchi}{\boldsymbol{\chi}}
\newcommand{\bLambda}{\boldsymbol{\Lambda}}
\newcommand{\bXi}{\boldsymbol{\Xi}}
\newcommand{\bxi}{\boldsymbol{\xi}}
\newcommand{\bvarsigma}{\boldsymbol{\varsigma}}
% my space-discretised variables
\newcommand{\buh}{\bu_h}
\newcommand{\bvh}{\bv_h}
\def\Dbuh{\boldsymbol{D}(\buh)}
\def\Dbvh{\boldsymbol{D}(\bvh)}
\newcommand{\ph}{p_h}
\newcommand{\qh}{q_h}
\newcommand{\gbuh}{\grad\bu_h}
\newcommand{\strh}{\strs_h}
% my regularized variables
\newcommand{\prd}{p_{\delta}}
\newcommand{\paL}{p_{\alpha}^L}
\newcommand{\padL}{p_{\alpha,\delta}^L}
\newcommand{\paorL}{p_{\alpha}^{(L)}}
\newcommand{\padorL}{p_{\alpha,\delta}^{(L)}}
\newcommand{\bud}{\bu_\delta}
\newcommand{\gbud}{\grad\bu_\delta}
\newcommand{\strd}{\strs_\delta}
\newcommand{\bua}{\bu_\alpha}
\newcommand{\gbua}{\grad\bu_\alpha}
\newcommand{\stra}{\strs_\alpha}
\newcommand{\budL}{\bu_{\delta,L}}
\newcommand{\buaL}{\bu_{\alpha}^{L}}
\newcommand{\buadL}{\bu_{\alpha,\delta}^{L}}
\newcommand{\buaorL}{\bu_{\alpha}^{(L)}}
\newcommand{\buadorL}{\bu_{\alpha,\delta}^{(L)}}
\newcommand{\gbudL}{\grad\bu_{\delta,L}}
\newcommand{\gbuaL}{\grad\bu_{\alpha}^{L}}
\newcommand{\gbuadL}{\grad\bu_{\alpha,\delta}^{L}}
\newcommand{\gbuaorL}{\grad\bu_{\alpha}^{(L)}}
\newcommand{\gbuadorL}{\grad\bu_{\alpha,\delta}^{(L)}}
\newcommand{\strdL}{\strs_{\delta,L}}
\newcommand{\straL}{\strs_{\alpha}^{L}}
\newcommand{\stradL}{\strs_{\alpha,\delta}^{L}}
\newcommand{\straorL}{\strs_{\alpha}^{(L)}}
\newcommand{\stradorL}{\strs_{\alpha,\delta}^{(L)}}
% my space-discretised regularized variables
\newcommand{\buhd}{\bu_{\delta,h}}
\newcommand{\gbuhd}{\grad\bu_{\delta,h}}
\newcommand{\strhd}{\strs_{\delta,h}}
\newcommand{\buhdL}{\bu_{\delta,L,h}}
\newcommand{\gbuhdL}{\grad\bu_{\delta,L,h}}
\newcommand{\stradh}{\strs_{\alpha,\delta,h}}
% my space-discretised regularized extra-diffusive variables
\newcommand{\buhda}{\bu_{\alpha,\delta,h}}
\newcommand{\gbuhda}{\grad\bu_{\alpha,\delta,h}}
\newcommand{\strhda}{\strs_{\alpha,\delta,h}}
\newcommand{\buhdLa}{\bu_{\alpha,\delta,h}}
\newcommand{\gbuhdLa}{\grad\bu_{\alpha,\delta,h}}
\newcommand{\strhdLa}{\strs_{\alpha,\delta,h}}
\newcommand{\buhLa}{\bu_{\alpha,h}}
\newcommand{\gbuhLa}{\grad\bu_{\alpha,h}}
\newcommand{\strhLa}{\strs_{\alpha,h}}
% my time-discretised variables
\newcommand{\strn}{\strs^n}
\newcommand{\strnp}{\strs^{n+1}}
\newcommand{\bun}{\bu^n}
\newcommand{\bunp}{\bu^{n+1}}
\newcommand{\pn}{p^n}
\newcommand{\qn}{q^n}
\newcommand{\Ln}{\Lambda^n}
\newcommand{\Lnp}{\Lambda^{n+1}}
% my space-and-time-discretised variables
\newcommand{\buhn}{\bu_h^n}
\newcommand{\buhnp}{\bu_h^{n+1}}
\newcommand{\buhnm}{\bu_h^{n-1}}
\def\Dbuhnm{\boldsymbol{D}(\buhnm)}
\def\Dbuhn{\boldsymbol{D}(\buhn)}
\def\Dbuhnp{\boldsymbol{D}(\buhnp)}
\newcommand{\strhnp}{\strs_h^{n+1}}
\newcommand{\strhnpp}{\strs_h^{n+1,+}}
\newcommand{\strhnpm}{\strs_h^{n+1,-}}
\newcommand{\strhn}{\strs_h^n}
\newcommand{\phn}{p_h^n}
\newcommand{\phnp}{p_h^{n+1}}
\newcommand{\gbuhnp}{\gbu_h^{n+1}}
\newcommand{\gbuhn}{\gbu_h^n}
%%%%%%%%%%%%%%%%%%%%%%%%%%%%%%%%
%\usepackage{verbatim}
%%
%% Pour activer les commentaires
%%
%%
\newcommand{\comment}[1]{ { ***~{\bf #1}~***}}
%%
%% Pour desactiver les commentaires
%%
% \newcommand{\comment}[1]{ }
%%%%%%%%%%%%%%%%%%%%%%%%%%%%%%%%
%--------------------------------------------------------
% \journalname{myjournal} % SPRINGER Insert the name of "your journal"

\begin{document}

\title{ALE method for gravity waves}

%\thanks{Grants or other notes
%about the article that should go on the front page should be
%placed here. General acknowledgments should be placed at the end of the article.} % SPRINGER
%\subtitle{..} % SPRINGER
%\titlerunning{Reduced-Basis : Uncertain coefficients in elliptic PDEs}% SPRINGER

\author{SB
%S\'ebastien Boyaval
% \footnote{
% Laboratoire d'hydraulique Saint-Venant,\\ Universit\'e Paris-Est
% (Ecole des Ponts ParisTech) EDF R\&D,\\ 6 quai Watier, 78401 Chatou Cedex, France
% \\[2mm]       
% sebastien.boyaval@enpc.fr
% \\[2mm]
% Present address: EPFL
%}
}
%\authorrunning{Short form of author list} % SPRINGER if too long for running head

% \begin{history}
% \received{}
% \revised{}
% \end{history}
%\date{Received: date / Accepted: date} % SPRINGER
%\date{\today}

\maketitle

\begin{abstract}
We investigate the capabilities of the ALE method for the numerical simulation of gravity waves.
% \PACS{PACS code1 \and PACS code2 \and more}
% \subclass{MSC code1 \and MSC code2 \and more}
% \keywords{Finite element method, convergence analysis, existence of weak solutions.}%
\end{abstract}

%\subjclass{} %

% \keywords{Finite element method, convergence analysis, existence of weak solutions.}%
% \ccode{AMS Subject Classification: 35Q30, 65M12, 65M60, 76A10, 76M10, 82D60}

%\tableofcontents

%\begin{acknowledgements}
%If you'd like to thank anyone, place your comments here
%and remove the percent signs.
%\end{acknowledgements}

\section{A 2D Finite-Element scheme for free-surface Stokes flows}

\subsection{IBVP}

$$
\D(t) = \{ z\in(b(x),\eta(t,x)) \ x \in (0,1) \} \,.
$$
% \begin{equation}
% \label{eq:NS}
%  (\partial_t+\bu\cdot\grad)\bu + \grad p - \div(2\eta\Dbu) = \f \quad \div\bu = 0 \,,
% \end{equation}
\begin{equation}
\label{eq:Stokes}
 \partial_t \bu+ \grad p - \div(2\eta\Dbu) = \f \quad \div\bu = 0 \,,
\end{equation}
% Pour repr\'esenter des ph\'enom\`enes hydrodynamiques sur un domaine de taille maximale $L[m]$ o\`u la vitesse est d'ordre $U[ms^{-1}]$, on prendra
% $\eta\approx 10^{-6}U/L^2$, la viscosit\'e cin\'ematique de l'eau \'etant suppos\'ee constante ici, de l'ordre de $10^{-6}m^2s^{-1}$ comme dans l'oc\'ean.
%   En multipliant toutes les dimensions d'espace par L, de temps par T et toutes les composantes de la solution par U=L/T, il suffit en effet de connaitre les solutions de
%   $$ \partial_t\bu + \grad p - \div(2\eta\Dbu) = 0 \quad \div\bu = 0 $$
%   dans tous les domaines bornés de taille 1 dans la direction bornée, pour toutes les conditions initiales de norme 1 et tous les nombres de Reynolds \eta
%   pour reconstruire ensuite toutes les solutions: il suffit de dilater correctement le temps et de conserver la similitude de Reynolds
\begin{equation}
\label{eq:impermeabilite}
 u|_{z=b(x,y)}\partial_x b %+ v|_{z=b(x,y)}\partial_y \eta
 - w|_{z=b(x,y)} = 0 \,,
\end{equation}
\begin{equation}
\label{eq:conditioncinematique}
 \partial_t\eta + u|_{z=\eta(t,x,y)}\partial_x \eta %+ v|_{z=\eta(t,x,y)}\partial_y \eta
 - w|_{z=\eta(t,x,y)} = 0 \,.
\end{equation}

$\bu=\grad\phi$ ($\grad\times\bu=0$) % Laplace-Bernoulli
\begin{equation}
 \label{eq:laplacebernoulli}
 \div \phi = 0 \quad \partial_t \phi + \frac{|\grad \phi|^2}2 + p - \f\cdot\xx = 0
\end{equation}
(\'eventuellement des conditions aux parois lat\'erales du domaine) et une condition initiale pour $(\phi,\eta)$, le syst\`eme dit ``des vagues''
\begin{equation}
 \label{eq:waterwave}
 \div \phi = 0 \quad \partial_t \phi|_{z=\eta(t,x,y)} + \frac{|\grad \phi|^2|_{z=\eta(t,x,y)}}2 + p_0 - \f\cdot\xx|_{z=\eta(t,x,y)} = 0
\end{equation}

\begin{equation}
\label{eq:vorticite}
 (\partial_t+\bu\cdot\grad)\omega - \omega\cdot\grad\bu = \nu \Delta \omega\,,
\end{equation}

\subsection{Scheme}

$([\mathbb{P}^k]^d,\mathbb{P}^{k-1})$

$$
\int_{\Omega} \left( \partial_t\bu\cdot\bv %+ [(\bu\cdot\grad)\bu]\cdot\bv 
- p \div\bv + q \div\bu + 2\eta\Dbu:\Dbv - \f\cdot\bv \right)
= 0  \quad  \forall (\bv,q)\in([\mathbb{P}^k]^d,\mathbb{P}^{k-1})
$$

$$
\frac{d}{dt} \int_{\Omega} \bu\cdot\bv + \int_{\Omega} \left( [(-\bw\cdot\grad)\bu]\cdot\bv % \bu
- \frac12(\div\bw)\bu\cdot\bv - p \div\bv + q \div\bu + 2\eta\Dbu:\Dbv - \f\cdot\bv \right)
= 0  \,. %\quad  \forall (\bv,q)\in([\mathbb{P}^k]^d,\mathbb{P}^{k-1})
$$

\begin{multline}
 \int_{\Omega^{n+1}} \bu^{n+1}\cdot\bv  - \int_{\Omega^{n}} \bu^{n}\cdot\bv  
+ \dt \int_{\Omega^{n+1}} \left( [(-\bw^{n}\circ(\I+\dt\bw^n)\cdot\grad)\bu^{n+1}]\cdot\bv %\bu^{n}
- \frac12(\div\bw^n)\bu^{n+1}\cdot\bv \right)
\\
+ \dt \int_{\Omega^{n+1}} \left( - p^{n+1} \div\bv + q \div\bu^{n+1} + 2\eta\Dbu^{n+1}:\Dbv - \f\cdot\bv \right) = 0 \quad \forall (\bv,q)\in([\mathbb{P}^k]^d,\mathbb{P}^{k-1})
\end{multline}
$\Omega^{n}\to\Omega^{n+1}:\xx\to\xx+\dt\bw^n(\xx)$

\begin{multline}
 \int_{\Omega^{n+1}} \left( \bu^{n+1}\cdot\bv - [\bu^{n}\circ(\I+\dt\bw^n)]\cdot\bv + \dt [((\bu^{n}-\bw^{n})\circ(\I+\dt\bw^n)\cdot\grad)\bu^{n+1}]\cdot\bv \right)
\\
+ \dt \int_{\Omega^{n+1}} \left( - p^{n+1} \div\bv + q \div\bu^{n+1} + 2\eta\Dbu^{n+1}:\Dbv - \f\cdot\bv \right) = 0 \quad \forall (\bv,q)\in([\mathbb{P}^k]^d,\mathbb{P}^{k-1})
\end{multline}

%\bibliographystyle{ws-m3as}
\bibliographystyle{amsplain}
%\bibliographystyle{plain}
%\bibliographystyle{plain}
%\bibliographystyle{spbasic}      % basic style, author-year citations
%\bibliographystyle{spmpsci}      % mathematics and physical sciences
%\bibliographystyle{spphys}       % APS-like style for physics
%\bibliography{/home/sebastien/Bureau/Z_boyaval/database/myrefs} % portable
\bibliography{/local00/home/SB03743S/Desktop/Z/database/myrefs} % EDF

\end{document}
